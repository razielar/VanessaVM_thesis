\section[Anthropogenic mass]{Anthropogenic mass}
\label{sec:anthropogenic_mass}

\begin{figure}[ht!]
  \centering
  \includegraphics[scale=0.19]{plots/introduction/mapsIntro.png}
  \caption[Worldwide urbanization evolution map between 1900 and 2050]{\textbf{Worldwide urbanization evolution map between 1900 and 2050}. Majority rural indicates less than 50\% of the population lives in urban areas, whereas majority urban indicates more than 50\%. Data obtained from.\autocite{ourworldindata_2020}}
  \label{fig:ghg}
\end{figure}




\begin{table}[!htb]
  \caption[Human Made Mass by years]{\textbf{Human Made Mass by years}. All human made mass are represented in mass by gigatons (m/Gt) from 1900 to 2020, divided into material groups. Human MM denotes human made mass. Data obtained from.\autocite{krausmann2017global}}
  \begin{scriptsize}
    \begin{tabulary}{0.65\linewidth}{clcccc}
      \textbf{Human MM} & \textbf{Description} & \textbf{1900 (m/Gt)} & \textbf{1940 (m/Gt)} & \textbf{1980 (m/Gt)} & \textbf{2020 (m/Gt)} \\ \hline
      Concrete & For building \& infrastructure & 2 & 10 & 86 & 549  \\
      Aggregates & Gravel \& sand & 17 & 30 & 135 & 386  \\
      Bricks & For construction & 11 & 16 & 28 & 92  \\
      Asphalt & For road \& pavement & 0 & 1 & 22 & 65  \\
      Metals & Iron, aluminum \& copper & 1 & 3 & 13 & 39  \\
      Other & Wood, paper, glass \& plastic & 4 & 6 & 11 & 23  \\
    \end{tabulary}
  \end{scriptsize}
  \label{tab:hmm_table}
\end{table}

\section[Energy sources]{Energy sources}
\label{sec:energy_sources}


