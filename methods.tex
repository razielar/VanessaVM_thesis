\section{Methods}
In this thesis, we attempt to explain both the growth of urbanization from the 1900s to 2000s as context for the decarbonization roadmaps that construction materials offer to reach the Net Zero goal of the Paris Agreement in 2050. Consequently, data on the prevalence of decarbonization roadmaps from the state-of-art materials was analyzed using a school assessment model, which assesses the impact of units  (\textit{e.g.} researchers, institutions, countries, publications, and sources) in three main metrics of productivity: the number of publications, citation count, and h-index (defined as the number of publications of an author/journal). To complete the collection of keywords, a thorough literature review was done in order to find various definitions and classifications.

\subsection{Scope of the study}

\begin{itemize}
\item As the country with the most information and where the changes caused by its geopolitical and economic hegemonization during the last century occurred, the US will be the primary focus of the discussion regarding technological development.
\item When information is accessible, data for building materials will be used, despite the fact that most of the research found has produced global numbers.
\end{itemize}


