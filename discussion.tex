\section{Challenges towards the construction sector’s decarbonization}
\label{sec:discussion_sec}

As urbanization sped up, production has increased, breaking the boundaries that constrained traditional settlements; including the widespread use of electricity in the 1930s, affecting how materials and buildings were assembled and installed; the standardization of steel, concrete, and glass products into the building sector as these materials were exploited and world trade increased; and the incorporation of polymers into construction allowed for the creation of custom materials and products. 

During the last century, it was widely believed that mass production would practically satisfy all needs. In order to achieve mass production, the government needed to actively engage in economic matters and support research and development, resulting in improvement initiatives like public housing, urban development, and transportation\autocite{harwood1969emergence}. These global events, such as wars and agreements, have triggered architectural movements.

Our work brings novelty to the literature by analyzing modern building materials from their global commercialization during the 20th century's industrialized urbanization era until the 2050 decarbonization target as a response to the Paris Agreement, introducing the decarbonization pathways, also known as roadmaps. These roadmaps concentrate on the emissions embodied in the structural (steel and concrete) and operational (plastic) materials. 

In addition to energy optimization, structural materials—steel and concrete—industries have reduced their emissions in the short to medium terms, primarily through increasing the reuse of ferrous scrap and clinker substitution, respectively. Addressing the efforts in all planning processes and among all stakeholders involved in decision-making at the various stages of a project—financing, designing, material production, construction and operation—to: avoid building (where possible), optimizing material use, recycling building materials and components, shifting to low-CO$_2$ materials and services. These actions require the least investment since the existing infrastructure is not remarkably modified, and have few impact on emissions reduction. Conversely, the decarbonization pathways rely heavily on CCS infrastructure, hydrogen storage, renewable electricity supply, and electricity grid expansion to achieve Net Zero in 2050. Greater economic investment as well as the utilization of minerals and raw materials are required by these actions.

This research shows that operational material—plastic—satisfies suppliers' interests in initial cost (although plastic components may need to be replaced more frequently than non-plastic alternatives) and requires less specialized knowledge, making it easier to install. However, the needs of the other stakeholders as well as the environmental impact must be taken into account. The construction industry is known to be highly fragmented, conservative and slow to adapt, hampering the spread of new technologies by a lack of knowledge transfer. The lack of knowledge regarding the real and current state of the materials embedded in buildings makes carrying out individual strategies difficult. 

The challenge of meeting climate targets is not only a technological one, but also one of economics and financial risk, especially given that the current climate policy is too weak. This is because there are already known measures and technologies that can reduce emissions to zero, including those that promote material efficiency and circularity, substitute biofuels or biomaterials, carbon capture and storage, and/or electrification with renewable electricity. To gain confidence in technologies, gain experience, and lower financial risk, large-scale demonstrations of key processes are in fact necessary, and highly mature technologies are already available. \autocite{karlsson2020roadmap}


